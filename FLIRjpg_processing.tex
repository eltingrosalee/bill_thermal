% Options for packages loaded elsewhere
\PassOptionsToPackage{unicode}{hyperref}
\PassOptionsToPackage{hyphens}{url}
%
\documentclass[
]{article}
\usepackage{amsmath,amssymb}
\usepackage{lmodern}
\usepackage{ifxetex,ifluatex}
\ifnum 0\ifxetex 1\fi\ifluatex 1\fi=0 % if pdftex
  \usepackage[T1]{fontenc}
  \usepackage[utf8]{inputenc}
  \usepackage{textcomp} % provide euro and other symbols
\else % if luatex or xetex
  \usepackage{unicode-math}
  \defaultfontfeatures{Scale=MatchLowercase}
  \defaultfontfeatures[\rmfamily]{Ligatures=TeX,Scale=1}
\fi
% Use upquote if available, for straight quotes in verbatim environments
\IfFileExists{upquote.sty}{\usepackage{upquote}}{}
\IfFileExists{microtype.sty}{% use microtype if available
  \usepackage[]{microtype}
  \UseMicrotypeSet[protrusion]{basicmath} % disable protrusion for tt fonts
}{}
\makeatletter
\@ifundefined{KOMAClassName}{% if non-KOMA class
  \IfFileExists{parskip.sty}{%
    \usepackage{parskip}
  }{% else
    \setlength{\parindent}{0pt}
    \setlength{\parskip}{6pt plus 2pt minus 1pt}}
}{% if KOMA class
  \KOMAoptions{parskip=half}}
\makeatother
\usepackage{xcolor}
\IfFileExists{xurl.sty}{\usepackage{xurl}}{} % add URL line breaks if available
\IfFileExists{bookmark.sty}{\usepackage{bookmark}}{\usepackage{hyperref}}
\hypersetup{
  pdftitle={FLIRjpg\_processing},
  pdfauthor={Rosalee Elting},
  hidelinks,
  pdfcreator={LaTeX via pandoc}}
\urlstyle{same} % disable monospaced font for URLs
\usepackage[margin=1in]{geometry}
\usepackage{color}
\usepackage{fancyvrb}
\newcommand{\VerbBar}{|}
\newcommand{\VERB}{\Verb[commandchars=\\\{\}]}
\DefineVerbatimEnvironment{Highlighting}{Verbatim}{commandchars=\\\{\}}
% Add ',fontsize=\small' for more characters per line
\usepackage{framed}
\definecolor{shadecolor}{RGB}{248,248,248}
\newenvironment{Shaded}{\begin{snugshade}}{\end{snugshade}}
\newcommand{\AlertTok}[1]{\textcolor[rgb]{0.94,0.16,0.16}{#1}}
\newcommand{\AnnotationTok}[1]{\textcolor[rgb]{0.56,0.35,0.01}{\textbf{\textit{#1}}}}
\newcommand{\AttributeTok}[1]{\textcolor[rgb]{0.77,0.63,0.00}{#1}}
\newcommand{\BaseNTok}[1]{\textcolor[rgb]{0.00,0.00,0.81}{#1}}
\newcommand{\BuiltInTok}[1]{#1}
\newcommand{\CharTok}[1]{\textcolor[rgb]{0.31,0.60,0.02}{#1}}
\newcommand{\CommentTok}[1]{\textcolor[rgb]{0.56,0.35,0.01}{\textit{#1}}}
\newcommand{\CommentVarTok}[1]{\textcolor[rgb]{0.56,0.35,0.01}{\textbf{\textit{#1}}}}
\newcommand{\ConstantTok}[1]{\textcolor[rgb]{0.00,0.00,0.00}{#1}}
\newcommand{\ControlFlowTok}[1]{\textcolor[rgb]{0.13,0.29,0.53}{\textbf{#1}}}
\newcommand{\DataTypeTok}[1]{\textcolor[rgb]{0.13,0.29,0.53}{#1}}
\newcommand{\DecValTok}[1]{\textcolor[rgb]{0.00,0.00,0.81}{#1}}
\newcommand{\DocumentationTok}[1]{\textcolor[rgb]{0.56,0.35,0.01}{\textbf{\textit{#1}}}}
\newcommand{\ErrorTok}[1]{\textcolor[rgb]{0.64,0.00,0.00}{\textbf{#1}}}
\newcommand{\ExtensionTok}[1]{#1}
\newcommand{\FloatTok}[1]{\textcolor[rgb]{0.00,0.00,0.81}{#1}}
\newcommand{\FunctionTok}[1]{\textcolor[rgb]{0.00,0.00,0.00}{#1}}
\newcommand{\ImportTok}[1]{#1}
\newcommand{\InformationTok}[1]{\textcolor[rgb]{0.56,0.35,0.01}{\textbf{\textit{#1}}}}
\newcommand{\KeywordTok}[1]{\textcolor[rgb]{0.13,0.29,0.53}{\textbf{#1}}}
\newcommand{\NormalTok}[1]{#1}
\newcommand{\OperatorTok}[1]{\textcolor[rgb]{0.81,0.36,0.00}{\textbf{#1}}}
\newcommand{\OtherTok}[1]{\textcolor[rgb]{0.56,0.35,0.01}{#1}}
\newcommand{\PreprocessorTok}[1]{\textcolor[rgb]{0.56,0.35,0.01}{\textit{#1}}}
\newcommand{\RegionMarkerTok}[1]{#1}
\newcommand{\SpecialCharTok}[1]{\textcolor[rgb]{0.00,0.00,0.00}{#1}}
\newcommand{\SpecialStringTok}[1]{\textcolor[rgb]{0.31,0.60,0.02}{#1}}
\newcommand{\StringTok}[1]{\textcolor[rgb]{0.31,0.60,0.02}{#1}}
\newcommand{\VariableTok}[1]{\textcolor[rgb]{0.00,0.00,0.00}{#1}}
\newcommand{\VerbatimStringTok}[1]{\textcolor[rgb]{0.31,0.60,0.02}{#1}}
\newcommand{\WarningTok}[1]{\textcolor[rgb]{0.56,0.35,0.01}{\textbf{\textit{#1}}}}
\usepackage{graphicx}
\makeatletter
\def\maxwidth{\ifdim\Gin@nat@width>\linewidth\linewidth\else\Gin@nat@width\fi}
\def\maxheight{\ifdim\Gin@nat@height>\textheight\textheight\else\Gin@nat@height\fi}
\makeatother
% Scale images if necessary, so that they will not overflow the page
% margins by default, and it is still possible to overwrite the defaults
% using explicit options in \includegraphics[width, height, ...]{}
\setkeys{Gin}{width=\maxwidth,height=\maxheight,keepaspectratio}
% Set default figure placement to htbp
\makeatletter
\def\fps@figure{htbp}
\makeatother
\setlength{\emergencystretch}{3em} % prevent overfull lines
\providecommand{\tightlist}{%
  \setlength{\itemsep}{0pt}\setlength{\parskip}{0pt}}
\setcounter{secnumdepth}{-\maxdimen} % remove section numbering
\ifluatex
  \usepackage{selnolig}  % disable illegal ligatures
\fi

\title{FLIRjpg\_processing}
\author{Rosalee Elting}
\date{3/4/2022}

\begin{document}
\maketitle

\textbf{IMAGE PROCESSING CODE} \emph{This code is designed to allow
users to process radiometric jpeg images produced from a FLIR camera
into raw data where each pixel is assigned a temperature. This workflow
was designed to then analyze these raw data files in ImageJ for
temperature analysis. The code is run in two chunks, with details
oulined below. ThermImage package was used for these analyses and
adapted to a loop. This package was created by Glenn Tattersall and can
be found \href{https://github.com/gtatters/Thermimage}{here}}

\textbf{CHUNK 1} \emph{This chunk will create a function that pulls jpgs
from an input folder and places the raw product into an output folder.
In Chunk 2, you will adapt the directories to your loacl machine where
you'd like these input and output folders. Before running this code you
will need to install \href{https://exiftool.org/install.htm}{Exiftools}
and follow the instructions below}

\textbf{MAC USERS}:\emph{In chunk 1 below, change any statement that
says ``exiftoolspath'' to read ``exiftoolpath=''installed``\emph{
\textbf{WINDOWS USERS}:}Save the downloaded ExifTools into your machines
C:/ folder. In chunk 1 below, change any statement that says''exiftools
path" to read ``exiftoolspath'''C:/". If you need to save it in another
location, that's okay, just reflect that in this directory.} \textbf{You
can do either of the code replacements above with the Ctrl/Cmd + F
function in R and replace with your needed path. Code will not run if R
cannot find your local Exiftools} \emph{Things to change in this chunk:
Exiftools location}

\begin{Shaded}
\begin{Highlighting}[]
\NormalTok{image\_processing }\OtherTok{\textless{}{-}} \ControlFlowTok{function}\NormalTok{(input,output) \{}
  \FunctionTok{require}\NormalTok{(Thermimage);}\FunctionTok{require}\NormalTok{(ggplot2);}\FunctionTok{require}\NormalTok{(tidyr);}\FunctionTok{require}\NormalTok{(Thermimage);}\FunctionTok{require}\NormalTok{(fields)}
\NormalTok{  images }\OtherTok{\textless{}{-}} \FunctionTok{list.files}\NormalTok{(}\AttributeTok{path=}\NormalTok{input, }\AttributeTok{pattern=} \StringTok{"*.jpg"}\NormalTok{)}
  \CommentTok{\#images \textless{}{-} list.files(path="/Users/kevinl.epperly/Dropbox/R\_scripts/bill\_thermal{-}main/input", pattern= "*.jpg")}
\NormalTok{  names }\OtherTok{\textless{}{-}} \FunctionTok{unique}\NormalTok{(images)}
\NormalTok{  nimages }\OtherTok{\textless{}{-}} \FunctionTok{length}\NormalTok{(images)}
  \ControlFlowTok{for}\NormalTok{ (i }\ControlFlowTok{in} \DecValTok{1}\SpecialCharTok{:}\NormalTok{nimages)\{}
    \CommentTok{\#set working directory to makes sure the code below is using the images in the input folder}
    \FunctionTok{setwd}\NormalTok{(input)}
    \CommentTok{\#read Flir jpg into img}
\NormalTok{    img}\OtherTok{\textless{}{-}}\FunctionTok{readflirJPG}\NormalTok{(names[i], }\AttributeTok{exiftoolpath=}\StringTok{"C:/"}\NormalTok{)}
    \CommentTok{\#get img dimensions}
\NormalTok{    dim }\OtherTok{\textless{}{-}} \FunctionTok{dim}\NormalTok{(img)}
    \CommentTok{\#get camera settings (cams) with flir settings}
\NormalTok{    cams }\OtherTok{\textless{}{-}}\FunctionTok{flirsettings}\NormalTok{(names[i], }\AttributeTok{exiftoolpath=}\StringTok{"C:/"}\NormalTok{, }\AttributeTok{camvals=}\StringTok{""}\NormalTok{)}
    \CommentTok{\#save heading of camsinfo}
\NormalTok{    head }\OtherTok{\textless{}{-}} \FunctionTok{head}\NormalTok{(}\FunctionTok{cbind}\NormalTok{(cams}\SpecialCharTok{$}\NormalTok{Info), }\DecValTok{20}\NormalTok{)}
    \CommentTok{\#save plancks from settings }
\NormalTok{    plancks}\OtherTok{\textless{}{-}}\FunctionTok{flirsettings}\NormalTok{(names[i], }\AttributeTok{exiftoolpath=}\StringTok{"C:/"}\NormalTok{, }\AttributeTok{camvals=}\StringTok{"{-}*Planck*"}\NormalTok{)}
    \FunctionTok{unlist}\NormalTok{(plancks}\SpecialCharTok{$}\NormalTok{Info)}
    \FunctionTok{cbind}\NormalTok{(}\FunctionTok{unlist}\NormalTok{(cams}\SpecialCharTok{$}\NormalTok{Dates))}
    \CommentTok{\#assign variables information in camsinfo}
    
\NormalTok{    ObjectEmissivity}\OtherTok{\textless{}{-}}\NormalTok{  cams}\SpecialCharTok{$}\NormalTok{Info}\SpecialCharTok{$}\NormalTok{Emissivity              }\CommentTok{\# Image Saved Emissivity {-} should be \textasciitilde{}0.95 or 0.96}
\NormalTok{    dateOriginal}\OtherTok{\textless{}{-}}\NormalTok{cams}\SpecialCharTok{$}\NormalTok{Dates}\SpecialCharTok{$}\NormalTok{DateTimeOriginal             }\CommentTok{\# Original date/time extracted from file}
\NormalTok{    dateModif}\OtherTok{\textless{}{-}}\NormalTok{   cams}\SpecialCharTok{$}\NormalTok{Dates}\SpecialCharTok{$}\NormalTok{FileModificationDateTime     }\CommentTok{\# Modification date/time extracted from file}
\NormalTok{    PlanckR1}\OtherTok{\textless{}{-}}\NormalTok{    cams}\SpecialCharTok{$}\NormalTok{Info}\SpecialCharTok{$}\NormalTok{PlanckR1                      }\CommentTok{\# Planck R1 constant for camera  }
\NormalTok{    PlanckB}\OtherTok{\textless{}{-}}\NormalTok{     cams}\SpecialCharTok{$}\NormalTok{Info}\SpecialCharTok{$}\NormalTok{PlanckB                       }\CommentTok{\# Planck B constant for camera  }
\NormalTok{    PlanckF}\OtherTok{\textless{}{-}}\NormalTok{     cams}\SpecialCharTok{$}\NormalTok{Info}\SpecialCharTok{$}\NormalTok{PlanckF                       }\CommentTok{\# Planck F constant for camera}
\NormalTok{    PlanckO}\OtherTok{\textless{}{-}}\NormalTok{     cams}\SpecialCharTok{$}\NormalTok{Info}\SpecialCharTok{$}\NormalTok{PlanckO                       }\CommentTok{\# Planck O constant for camera}
\NormalTok{    PlanckR2}\OtherTok{\textless{}{-}}\NormalTok{    cams}\SpecialCharTok{$}\NormalTok{Info}\SpecialCharTok{$}\NormalTok{PlanckR2                      }\CommentTok{\# Planck R2 constant for camera}
\NormalTok{    ATA1}\OtherTok{\textless{}{-}}\NormalTok{        cams}\SpecialCharTok{$}\NormalTok{Info}\SpecialCharTok{$}\NormalTok{AtmosphericTransAlpha1        }\CommentTok{\# Atmospheric Transmittance Alpha 1}
\NormalTok{    ATA2}\OtherTok{\textless{}{-}}\NormalTok{        cams}\SpecialCharTok{$}\NormalTok{Info}\SpecialCharTok{$}\NormalTok{AtmosphericTransAlpha2        }\CommentTok{\# Atmospheric Transmittance Alpha 2}
\NormalTok{    ATB1}\OtherTok{\textless{}{-}}\NormalTok{        cams}\SpecialCharTok{$}\NormalTok{Info}\SpecialCharTok{$}\NormalTok{AtmosphericTransBeta1         }\CommentTok{\# Atmospheric Transmittance Beta 1}
\NormalTok{    ATB2}\OtherTok{\textless{}{-}}\NormalTok{        cams}\SpecialCharTok{$}\NormalTok{Info}\SpecialCharTok{$}\NormalTok{AtmosphericTransBeta2         }\CommentTok{\# Atmospheric Transmittance Beta 2}
\NormalTok{    ATX}\OtherTok{\textless{}{-}}\NormalTok{         cams}\SpecialCharTok{$}\NormalTok{Info}\SpecialCharTok{$}\NormalTok{AtmosphericTransX             }\CommentTok{\# Atmospheric Transmittance X}
\NormalTok{    OD}\OtherTok{\textless{}{-}}\NormalTok{          cams}\SpecialCharTok{$}\NormalTok{Info}\SpecialCharTok{$}\NormalTok{ObjectDistance                }\CommentTok{\# object distance in metres}
\NormalTok{    FD}\OtherTok{\textless{}{-}}\NormalTok{          cams}\SpecialCharTok{$}\NormalTok{Info}\SpecialCharTok{$}\NormalTok{FocusDistance                 }\CommentTok{\# focus distance in metres}
\NormalTok{    ReflT}\OtherTok{\textless{}{-}}\NormalTok{       cams}\SpecialCharTok{$}\NormalTok{Info}\SpecialCharTok{$}\NormalTok{ReflectedApparentTemperature  }\CommentTok{\# Reflected apparent temperature}
\NormalTok{    AtmosT}\OtherTok{\textless{}{-}}\NormalTok{      cams}\SpecialCharTok{$}\NormalTok{Info}\SpecialCharTok{$}\NormalTok{AtmosphericTemperature        }\CommentTok{\# Atmospheric temperature}
\NormalTok{    IRWinT}\OtherTok{\textless{}{-}}\NormalTok{      cams}\SpecialCharTok{$}\NormalTok{Info}\SpecialCharTok{$}\NormalTok{IRWindowTemperature           }\CommentTok{\# IR Window Temperature}
\NormalTok{    IRWinTran}\OtherTok{\textless{}{-}}\NormalTok{   cams}\SpecialCharTok{$}\NormalTok{Info}\SpecialCharTok{$}\NormalTok{IRWindowTransmission          }\CommentTok{\# IR Window transparency}
\NormalTok{    RH}\OtherTok{\textless{}{-}}\NormalTok{          cams}\SpecialCharTok{$}\NormalTok{Info}\SpecialCharTok{$}\NormalTok{RelativeHumidity              }\CommentTok{\# Relative Humidity}
\NormalTok{    h}\OtherTok{\textless{}{-}}\NormalTok{           cams}\SpecialCharTok{$}\NormalTok{Info}\SpecialCharTok{$}\NormalTok{RawThermalImageHeight         }\CommentTok{\# sensor height (i.e. image height)}
\NormalTok{    w}\OtherTok{\textless{}{-}}\NormalTok{           cams}\SpecialCharTok{$}\NormalTok{Info}\SpecialCharTok{$}\NormalTok{RawThermalImageWidth          }\CommentTok{\# sensor width (i.e. image width)}
    
    \CommentTok{\#create a sting from image }
    \FunctionTok{str}\NormalTok{(img)}
    \CommentTok{\#make data frame of temperature from raw data and show resulting string}
\NormalTok{    temperature}\OtherTok{\textless{}{-}}\FunctionTok{raw2temp}\NormalTok{(img, ObjectEmissivity, OD, ReflT, AtmosT, IRWinT, IRWinTran, RH,}
\NormalTok{                          PlanckR1, PlanckB, PlanckF, PlanckO, PlanckR2, }
\NormalTok{                          ATA1, ATA2, ATB1, ATB2, ATX)}
    \FunctionTok{str}\NormalTok{(temperature)}
    \CommentTok{\#plot temperature data, can add rotation note here (ie. trans="rotate270.matrix") or the color palette (ie.thermal.palette=rainbowpal)}
    
    \FunctionTok{plotTherm}\NormalTok{(temperature, }\AttributeTok{h=}\NormalTok{h, }\AttributeTok{w=}\NormalTok{w, }\AttributeTok{minrangeset=}\DecValTok{21}\NormalTok{, }\AttributeTok{maxrangeset=}\DecValTok{32}\NormalTok{)}
    \CommentTok{\#set new WD for the file, since I\textquotesingle{}d like it save in outputs}
    \FunctionTok{setwd}\NormalTok{(output)}
    \CommentTok{\#setwd("/Users/kevinl.epperly/Dropbox/R\_scripts/bill\_thermal{-}main/output")}
    \CommentTok{\#Write the raw folder into the outpur folder.}
    \FunctionTok{writeFlirBin}\NormalTok{(}\FunctionTok{as.vector}\NormalTok{(}\FunctionTok{t}\NormalTok{(temperature)), }\AttributeTok{templookup=}\ConstantTok{NULL}\NormalTok{, }\AttributeTok{w=}\NormalTok{w, }\AttributeTok{h=}\NormalTok{h, }\AttributeTok{I=}\StringTok{""}\NormalTok{, }\AttributeTok{rootname=}\NormalTok{cams}\SpecialCharTok{$}\NormalTok{Info}\SpecialCharTok{$}\NormalTok{FileName)}
\NormalTok{  \} }
\NormalTok{\}}
\end{Highlighting}
\end{Shaded}

\textbf{CHUNK 2} \emph{This Chunk will run the function we made in chunk
1 called} \textbf{Image processing} \emph{This will pull the input from
images that you have saved and place them in an output folder. Therfore
for this chunk you will need to put your machine's directory into the
argument below. They can be identical, however i recommend making
subfolers within your project titled ``input'' and ``output''. You will
use their location in the path below for your machine.} \emph{Thinks to
change in this chunk: the paths of your input and output on your local
machine}

\begin{Shaded}
\begin{Highlighting}[]
\CommentTok{\#The format for this code is image\_processing("input location", "output location")}
\CommentTok{\#Sample: image\_processing("C:/Users/Mellisuga/Documents/R/bill\_thermal/input","C:/Users/Mellisuga/Documents/R/bill\_thermal/output")}
\FunctionTok{image\_processing}\NormalTok{(}\StringTok{"C:/Users/Mellisuga/Documents/R/bill\_thermal/input"}\NormalTok{,}\StringTok{"C:/Users/Mellisuga/Documents/R/bill\_thermal/output"}\NormalTok{)}
\end{Highlighting}
\end{Shaded}

\begin{verbatim}
## Loading required package: Thermimage
\end{verbatim}

\begin{verbatim}
## Warning: package 'Thermimage' was built under R version 4.1.2
\end{verbatim}

\begin{verbatim}
## Loading required package: ggplot2
\end{verbatim}

\begin{verbatim}
## Loading required package: tidyr
\end{verbatim}

\begin{verbatim}
## Loading required package: fields
\end{verbatim}

\begin{verbatim}
## Warning: package 'fields' was built under R version 4.1.2
\end{verbatim}

\begin{verbatim}
## Loading required package: spam
\end{verbatim}

\begin{verbatim}
## Warning: package 'spam' was built under R version 4.1.2
\end{verbatim}

\begin{verbatim}
## Spam version 2.8-0 (2022-01-05) is loaded.
## Type 'help( Spam)' or 'demo( spam)' for a short introduction 
## and overview of this package.
## Help for individual functions is also obtained by adding the
## suffix '.spam' to the function name, e.g. 'help( chol.spam)'.
\end{verbatim}

\begin{verbatim}
## 
## Attaching package: 'spam'
\end{verbatim}

\begin{verbatim}
## The following objects are masked from 'package:base':
## 
##     backsolve, forwardsolve
\end{verbatim}

\begin{verbatim}
## Loading required package: viridis
\end{verbatim}

\begin{verbatim}
## Loading required package: viridisLite
\end{verbatim}

\begin{verbatim}
## 
## Try help(fields) to get started.
\end{verbatim}

\begin{verbatim}
##  int [1:480, 1:640] 11703 11701 11717 11702 11709 11713 11722 11723 11730 11734 ...
##  num [1:480, 1:640] 9.01 8.99 9.12 9 9.05 ...
\end{verbatim}

\includegraphics{FLIRjpg_processing_files/figure-latex/unnamed-chunk-2-1.pdf}

\begin{verbatim}
##  int [1:480, 1:640] 13632 13672 13746 13837 13931 13982 14093 14157 14165 14259 ...
##  num [1:480, 1:640] 23.4 23.7 24.2 24.8 25.5 ...
\end{verbatim}

\includegraphics{FLIRjpg_processing_files/figure-latex/unnamed-chunk-2-2.pdf}

\begin{verbatim}
##  int [1:480, 1:640] 13716 13700 13701 13694 13695 13696 13690 13713 13693 13692 ...
##  num [1:480, 1:640] 24 23.9 23.9 23.8 23.8 ...
\end{verbatim}

\includegraphics{FLIRjpg_processing_files/figure-latex/unnamed-chunk-2-3.pdf}

\textbf{Your raw files should now be available to view in your output
folder!}

\emph{This Rmd Can be knit into a HTML or PDF and will create a report
on your photos for future reference, should you choose}

\end{document}
